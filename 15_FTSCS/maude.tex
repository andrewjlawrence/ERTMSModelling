\section{Maude/Real-Time Maude}

The Maude system \cite{MC03} is a multi-purpose tool with support for
executable specification, simulation and verification.
%
%%and our approach uses all of these capabilities.
%
Its wide range of capabilities made us to favour Maude. Particularly,
we are interested in the Maude LTL Model Checker \cite{ES00}.  Real
Time Maude \cite{PO07b} is an extension of Maude containing specific
support enabling the modelling and verification
%
%of discrete time and 
%
real-time systems.
%
%The Real-Time Maude system contains models of discrete time based on the natural numbers and dense time based on the rational numbers.
%
%% Maude specifications are built from \emph{functional modules} 
%% %
%% %declared using \texttt{fmod} and \texttt{endfm}
%% %
%% which contain the following:
%% %
%% \begin{center}
%% \begin{tabular}{| c | l |}
%% \hline sorts & \texttt{sort} $s$ or \texttt{sorts} $s \ s' .$
%% \\ \hline subsorts & \texttt{subsort} $s < s' \ .$ \\ \hline function
%% symbols & \texttt{op} $f \ : \ s_1 \ldots s_n$ \texttt{->} $s \ .$
%% \\ \hline variables & \texttt{vars} $v \ v' : s' .$\\ \hline
%% unconditional equations &\texttt{eq} $t = t' .$\\ \hline conditional
%% equations & \texttt{ceq} $t = t'$ \texttt{if} $cond$ \\ \hline
%% membership axioms & \texttt{mb} $t \ : \ s \ .$ or \texttt{cmb} $t \ :
%% \ s$ \texttt{if} $cond \ .$ \\ \hline unconditional rules &
%% \texttt{rl} [label] : $t => t' .$ \\ \hline conditional rules &
%% \texttt{crl} [label] : $t => t'$ \texttt{if} $cond$ \\ \hline
%% \end{tabular}
%% \end{center}

Object-based systems can be modelled as multisets of objects and
messages where the messages define the communication between the
objects and typically trigger actions of the objects. A class $C$ with
attributes of \verb|a_1| to \verb|a_n| of sort \verb|Sort_1| to \verb|Sort_n|, and an object {\tt O}
with attribute values \verb|v_1| to \verb|v_n| of class {\tt C} are written as, respectively
%
\begin{lstlisting}[columns=fixed]
class C | a_1 : Sort_1, ... , a_n : Sort_n .  
< O : C | a_1 : v_1, ... , a_n : v_n > .
\end{lstlisting}
%
Objects declared together with messages
\begin{lstlisting}[columns=fixed]
msgs M_1 ... M_k : Sort_1 ... Sort_n -> Msg .  
\end{lstlisting}
form a multiset of the sort \texttt{Configuration}, a subsort of Maude's built-in sort
\texttt{System},  using \verb|__| for multiset union.
\begin{lstlisting}[columns=fixed]
sorts Object Msg Configuration .
subsort Object Msg < Configuration . 
op __ : Configuration Configuration -> Configuration [ctor] .
\end{lstlisting}

A real-time specification \cite{PO07b} consists of a sort \verb|Time|
(in our case \verb|PosRat|), the constructor
%
\verb| {_} : System -> Globalsystem|
%
with the meaning that \verb|{t}| represents the whole system (and does not appear
as an argument to another function - as is marked by using the independent type \texttt{Globalsystem}),
instantaneous rewrite rules, and a so-called tick rule that defines how time elapses.
%
As \cite{PO07}, we use the operators \verb|delta| and \verb|mte| in
order to define the effect of time elapse on a configuration, and of the
maximal possible time elapse, resp.
%
\begin{lstlisting}[columns=fixed]
op delta : Configuration Time -> Configuration [frozen (1)] . 
op mte : Configuration -> TimeInf [frozen (1)] .
\end{lstlisting}
Here, \texttt{TimeInf} is the sort \texttt{Time} enriched with an infinity element
\texttt{Inf}. These two functions are distributed
over objects and messages, i.e., each
object has the same time available, and as the maximal time elapse
for a message has value 0, time can only progress once all messages
are consumed.
%
\begin{lstlisting}[columns=fixed]
vars CON1 CON2 : NeConfiguration . var R : Time .
eq delta(none, R) = none .
eq delta(CON1 CON2, R) = delta(CON1,R) delta(CON2,R) . 
eq mte(none) = INF .
eq mte(CON1 CON2) = min(mte(CON1),mte(CON2)) .
\end{lstlisting}
%
The argument {\tt R} of type {\tt Time} is determined by the tick rule 
%
\begin{lstlisting}
crl [tick] : {CURRENT} => {delta(CURRENT,R)} in time R 
                if R <= mte(CURRENT) [nonexec] .
\end{lstlisting}
%
The default tick time is defined by
\begin{lstlisting}[columns=fixed]
(set tick def 1 .) 
\end{lstlisting}
%
This means we look at the configuration either at each time step, or
more often in the case that some event occurs, for a justification see
e.g.\ \cite{PO06}.


%% For a theoretical justification of this approach as well as completeness results see  e.g. \cite{PO06}.
%%Meseguer and O show in [completenesspaper] under which conditions
%%the specification is time robust and therefore complete.
%% In the next section we will give additional domain specific equations
%% for \verb|delta| and \verb|mte| which define the behaviour of all
%% objects involved.
