\section{Model Checking Results}
\label{sec:modelchecking}

In this section we verify a number of rail yards with the Real-Time
Maude Tool~\cite{olveczky2008real}. We check that that the invariant
``no collisions'', c.f.\ Section \ref{sec:safetycondmodelling}, is
globally true, either for all time 
\begin{lstlisting}[columns=fixed]
mc initState |=t [] nocrashDistance(train1,train2) . 
\end{lstlisting}
or for 300 time steps:
\begin{lstlisting}[columns=fixed]
mc initState |=t [] nocrashDistance(train1,train2) in time <= 300 . 
\end{lstlisting}

Here, \verb|initState| is as given in Section
\ref{sec:errorInjection}. As track plans, we consider the pass-through
station shown in Figure \ref{fig:station} as well as some variations
of it, see Figure \ref{fig:threestation}. This is in order to obtain
an indication of how variations in the complexity of the rail yard
influence the time required for model checking.

\begin{figure}[H]
\centering
\begin{tikzpicture}[scale=0.5,transform shape]
\draw [->] (0,-0.8) to (1,-0.8);

\RWConnector{a0}{(0,-2)}
\RWConnector{a1}{(2,-2)}
\RWLinearUnitBelow{a0}{a1}{Entry}
\RWSignal{MB1}{(2,-2)}{MB1}
\RWConnector{a2}{(4,-2)}
\RWLinearUnitBelow{a1}{a2}{AA(500m)}
\RWConnector{a3}{(6,-2)}
\RWLinearUnitBelow{a2}{a3}{AB(500m)}

\RWPoint{p101}{a4}{a5}{p101}{(7,-2)}
\RWLabelPointAboveLeft{a5}{a5}{P1}
\RWLabelLinearUnitBelow{a4}{a5}{AC(150m)}

\RWConnector{b6}{(10,-0.5)}
\RWLinearUnitBelow{p101}{b6}{BD(500m)}
\RWConnector{b7}{(12,-0.5)}
\RWLinearUnitBelow{b6}{b7}{BE(500m)}

\RWConnector{a6}{(10,-2)}
\RWLinearUnitBelow{a5}{a6}{AD(500m)}
\RWConnector{a7}{(12,-2)}
\RWLinearUnitBelow{a6}{a7}{AE(500m)}

\RWConnector{a8}{(14,-2)}
\RWLinearUnitBelow{a7}{a8}{Exit1}
\RWConnector{b8}{(14,-0.5)}
\RWLinearUnitBelow{b7}{b8}{Exit2}
\end{tikzpicture}
\quad \\
\begin{tikzpicture}[scale=0.5,transform shape]
\draw [->] (0,-0.8) to (1,-0.8);

\RWConnector{a0}{(0,-2)}
\RWConnector{a1}{(2,-2)}
\RWLinearUnitBelow{a0}{a1}{Entry}
\RWSignal{MB1}{(2,-2)}{MB1}
\RWConnector{a2}{(4,-2)}
\RWLinearUnitBelow{a1}{a2}{AA(1500m)}
\RWPoint{p101}{a3}{a4}{p101}{(5,-2)}
\RWLabelPointAboveLeft{a4}{a4}{P1}
\RWLabelLinearUnitBelow{a3}{a4}{AB(250m)}

\RWConnector{b5}{(8,-0.5)}
\RWLinearUnitBelow{p101}{b5}{BC(1500m)}

\RWPoint{p102}{b6}{b7}{p102}{(9,-0.5)}
\RWLabelPointAboveLeft{b7}{b7}{P2}
\RWLabelLinearUnitBelow{b6}{b7}{BD(250m)}

\RWConnector{b8}{(12,-0.5)}
\RWLinearUnitBelow{b7}{b8}{BE(1500m)}
%\RWSignal{MB2}{(9,-0.5)}{MB2}

\RWConnector{a5}{(8,-2)}
\RWLinearUnitBelow{a4}{a5}{AC(1500m)}
%\RWSignal{MB3}{(9,-2)}{MB3}
\RWConnector{a6}{(10,-2)}
\RWLinearUnitBelow{a5}{a6}{AD(1500m)}


\RWConnector{c7}{(12,1)}
\RWLinearUnitBelow{p102}{c7}{CE(1500m)}
\RWConnector{c8}{(14,1)}
\RWLinearUnitBelow{c7}{c8}{CF(1500m)}
\RWSignal{MB2}{(14,1)}{MB4}
\RWConnector{c9}{(16,1)}
\RWLinearUnitBelow{c8}{c9}{CG(1500m)}

\RWConnector{b9}{(14,-0.5)}
\RWLinearUnitBelow{b8}{b9}{BF(1500m)}
\RWConnector{b10}{(16,-0.5)}
\RWLinearUnitBelow{b9}{b10}{BG(1500m)}
\RWSignal{MB3}{(14,-0.5)}{MB5}

\RWConnector{a7}{(12,-2)}
\RWLinearUnitBelow{a6}{a7}{AE(1500m)}
\RWConnector{a8}{(14,-2)}
\RWLinearUnitBelow{a7}{a8}{AF(1500m)}
\RWConnector{a9}{(16,-2)}
\RWLinearUnitBelow{a8}{a9}{AG(1500m)}
\RWSignal{MB4}{(14,-2)}{MB6}
\RWConnector{a10}{(18,-2)}
\RWLinearUnitBelow{a9}{a10}{AH(1500m)}
\RWConnector{a11}{(20,-2)}
\RWLinearUnitBelow{a10}{a11}{AI(1500m)}

\RWPointReverse{p103}{b11}{b12}{p103}{(17,-0.5)}
\RWLabelPointAboveRight{b12}{b12}{P3}
\RWLabelLinearUnitBelow{b11}{b12}{BH(250m)}

\RWConnector{b13}{(20,-0.5)}
\RWLinearUnitBelow{b11}{b13}{BI(1500m)}

\RWPointReverse{p104}{a12}{a13}{p104}{(21,-2)}
\RWLabelPointAboveRight{a13}{a13}{P4}
\RWLabelLinearUnitBelow{a12}{a13}{AJ(250m)}

\RWConnector{a14}{(24,-2)}
\RWLinearUnitBelow{a12}{a14}{Exit}

\end{tikzpicture}
\caption{Track plans for a junction and three platform station.}
\label{fig:junction}
\label{fig:threestation}
\end{figure}

We check all three track plans with manually constructed tables that
we consider to be correct. In all settings the model checking confirms
that these rail yard designs are collision free (within the given
time-bound, if applicable).  The table shows verification
times\footnote{Using a PC running Xubuntu
  14.04.2 with an i7 4790 @3.60Ghz and 32GB RAM.} and the number of
rewrite steps for the three rail yards against the random controller
and the round-robin controller (see Section~\ref{ssec:instan}).  The
following table presents our model checking results.
%
%% Here, we follow the finitisation technique given by James et
%% al.~\cite{sttt14}, namely that proving collision-freedom for two
%% trains is enough for proving safety in general. However, this result
%% requires a proof on based around the encoding of our railway DSL in
%% real-time maude.  Even though our encoding follows James at al's, this
%% proof remains as future work.
%
{\small
  \begin{center}
  \begin{tabular}{|l|c|c|}
    \hline 
    \textbf{Scheme Plan} & \textbf{Round Robin Controller} &
    \textbf{Random Controller} \\
    & \textbf{Unbounded} & \textbf{In Time 300} \\
    \hline \hline
    Junction  & 2.4s / 5,767,435 rewrites & 268.3s / 208,715,358 rewrites \\
    Pass-through Station & 3.0s  / 7,135,987 rewrites & 439.2s / 308,629,500 rewrites \\
    Three Platform Station & 2.8s / 6,624,578 rewrites & 2697.1s / 729,201,878 rewrites \\
    \hline
 \end{tabular}
  \end{center}
}
%% \caption{Verification results of model checking three scheme plans.}
%% \label{tab:results}
%% \end{figure}
%

The table shows that unbounded model checking is successful when
control is restricted, e.g., to our round-robin controller. However,
when using our random controller, the state space vastly
increases. Thus, we provide results for up to a given time bound of
$300s$. Note that this time is enough to ensure that one train can
travel completely through the Junction and Station scheme
plan. Interestingly, the model checking times for the three platform
station illustrate that train movements across particular routes are
restricted by applying the round robin controller. This is due to the
number of routes that share tracks and hence impose restrictions on
each other when in use. This in turn means the model checking of the
three station takes less time than the station when using the round
robin control strategy, as there are less possible rewrites to be
considered. It is future work, to consider further, more varied rail
yards.
