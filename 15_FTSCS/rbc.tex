The RBC mediates between requests from the trains to extend their
movement authorities and the successful route requests by the
controller. To this end it reconciles two different views on the rail
yard: trains use continuous data to represent their position (in our
model the distance from the leftmost point of the rail yard); the
interlocking uses discrete data (track occupation, set routes, point
positions) in its logic.
%
In our model, we take a rather simplified and also abstract view on
the challenges involved. We make the assumption that trains request a
new movement authority only on the track on which their current
authority ends. Furthermore, we abstract the mapping between
continuous and discrete data to the two tables presented in Figure
\ref{fig:rbctables}.

In our model, the RBC only holds information on successful route
requests (in \verb|availableRoutes|) and for which trains
(characterised by their \verb|Oid|) it currently has an open ``request
to proceed'' (in \verb|designatedRoutes|):
%
\begin{lstlisting}[columns=fixed]
class RBC | availableRoutes  : SetOfRouteNames, 
            designatedRoutes : MapOid2RouteName  .
\end{lstlisting}

Also, the RBC is a passive system component. A typical reaction is the
following: When the interlocking sends a ``proceed message'' for a
route \verb|RN|, the RBC sends a new ``end of authority'' to the train
and removes the corresponding request from its internal state.
%
\begin{lstlisting}[columns=fixed]
eq   proceedgrant(RN) < O2 : RBC | designatedRoutes : TRN > 
   = magrant(getTrain(RN, TRN), endOfAuthority(RN)) 
     < O2 : RBC | designatedRoutes : removeRoute(_,_) >  . 
\end{lstlisting}
