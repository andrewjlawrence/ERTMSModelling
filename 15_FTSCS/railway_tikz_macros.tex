\usepackage{tikz}
\usetikzlibrary{arrows}
\usetikzlibrary{automata}
\usetikzlibrary{backgrounds}
\usetikzlibrary{calc}
\usetikzlibrary{fit}
\usetikzlibrary{shapes}
\usetikzlibrary{snakes}


\tikzset{sortTZ/.style ={}}
\tikzset{subsortTZ/.style = {draw, <-}}
\tikzset{isomorphicTZ/.style = {subsortTZ, <->}}
\tikzset{fixedTextHeightProblem/.style = {text height=1.5ex,text depth=.25ex}}

%%%%%%%%%%%%%%%%%%%%%%%%%%%%%%%%%%%%%%%%%%%%%%%%%%%%%%%%%%%%
%                                                          %
%                                                          %
% Railway macros                                           %
%                                                          %
%                                                          %
%%%%%%%%%%%%%%%%%%%%%%%%%%%%%%%%%%%%%%%%%%%%%%%%%%%%%%%%%%%%

%%%%%%%%%%%%%%%%%%%%%%%%%%%%%%%%%%%%%%%%%%%%%%%%%%%%%%%%%%%%
%% Customiseable Lengths
%%%%%%%%%%%%%%%%%%%%%%%%%%%%%%%%%%%%%%%%%%%%%%%%%%%%%%%%%%%%

\newcommand{\RWConnectorHalfHeight}{1mm}
\newcommand{\RWPointHeight}{15mm}
\newcommand{\RWPointHalfWidth}{10mm}
%% The lengths of the units attached to the point within the junction
\newcommand{\RWJunctionUnitLength}{10mm}
%% The gap vertical between the platform edge and the track
\newcommand{\RWPlatformGapFromTrack}{2mm}
%% The gap horizontal gap between the platform edge and the connector
\newcommand{\RWPlatformGapFromConnector}{5mm}
%% The vertical height of the platform
\newcommand{\RWPlatformHeight}{3mm}

\newcommand{\RWSignalHeight}{6mm}

\newcommand{\RWSemiLongPointHalfWidth}{20mm}
\newcommand{\RWLongPointHalfWidth}{30mm}

%%%%%%%%%%%%%%%%%%%%%%%%%%%%%%%%%%%%%%%%%%%%%%%%%%%%%%%%%%%%
%% Connectors
%%%%%%%%%%%%%%%%%%%%%%%%%%%%%%%%%%%%%%%%%%%%%%%%%%%%%%%%%%%%

% param 1 = name of connector (must not exist)
% param 2 = coordinate for center of connector
\newcommand{\RWConnector}[2]{
  \coordinate (#1) at #2;
  \draw ($(#1) +(0,-\RWConnectorHalfHeight)$) -- ($(#1) +(0,+\RWConnectorHalfHeight)$);
}

% param 1 = name of connector to draw at (must exist)
% param 2 = label to be used on connector
\newcommand{\RWLabelConnectorBelow}[2]{
  \node [anchor = north] at ($(#1) +(0,-\RWConnectorHalfHeight)$) {#2};
}

% param 1 = name of connector to label (must exist)
% param 2 = label to be used on connector
\newcommand{\RWLabelConnectorAbove}[2]{
b  \node [anchor = south] at ($(#1) +(0,+\RWConnectorHalfHeight)$) {#2};
}




% param 1 = name of connector 1 (must exist)
% param 2 = name of connector 2 (must exist)
% param 3 = label to be used on linear unit
\newcommand{\RWLabelPointAboveRight}[3]{
  \node [anchor = north] at ($(#1)!.5!(#2) +(0.4,6*\RWConnectorHalfHeight)$) {#3};
}


% param 1 = name of connector 1 (must exist)
% param 2 = name of connector 2 (must exist)
% param 3 = label to be used on linear unit
\newcommand{\RWLabelPointAboveLeft}[3]{
  \node [anchor = north] at ($(#1)!.5!(#2) +(-0.4,6*\RWConnectorHalfHeight)$) {#3};
}

% param 1 = name of connector 1 (must exist)
% param 2 = name of connector 2 (must exist)
% param 3 = label to be used on linear unit
\newcommand{\RWLabelPointAbove}[3]{
  \node [anchor = north] at ($(#1)!.5!(#2) +(0,6*\RWConnectorHalfHeight)$) {#3};
}

% param 1 = name of connector 1 (must exist)
% param 2 = name of connector 2 (must exist)
% param 3 = label to be used on linear unit
\newcommand{\RWLabelLinearUnitAbove}[3]{
  \node [anchor = north] at ($(#1)!.5!(#2) +(0,5.2*\RWConnectorHalfHeight)$) {#3};
}

% param 1 = name of connector 1 (must exist)
% param 2 = name of connector 2 (must exist)
% param 3 = label to be used on linear unit
\newcommand{\RWLabelLinearUnitBelow}[3]{
  \node [anchor = north] at ($(#1)!.5!(#2) +(0,-1.2*\RWConnectorHalfHeight)$) {#3};
}

% param 1 = name of connector 1 (must exist)
% param 2 = name of connector 2 (must exist)
% param 3 = label to be used on linear unit
\newcommand{\RWLinearUnitAbove}[3] {
  \draw (#1) -- (#2);
  \RWLabelLinearUnitAbove{#1}{#2}{#3}
}
% param 1 = name of connector 1 (must exist)
% param 2 = name of connector 2 (must exist)
% param 3 = label to be used on linear unit
\newcommand{\RWLinearUnitBelow}[3] {
  \draw (#1) -- (#2);
  \RWLabelLinearUnitBelow{#1}{#2}{#3}
}

% param 1 = name of connector 1 (must exist)
% param 2 = name of connector 2 (must exist)
% param 3 = label to be used on linear unit
\newcommand{\RWOverlapAbove}[3] {
  \draw (#1) -- (#2);
 %% \coordinate (#1) at #2;
  \draw ($(#2) -(0.1,-\RWConnectorHalfHeight)$) -- ($(#2) -(0,-\RWConnectorHalfHeight)$);
  \draw ($(#2) -(0.1,+\RWConnectorHalfHeight)$) -- ($(#2) -(0,+\RWConnectorHalfHeight)$);
  \RWLabelLinearUnitAbove{#1}{#2}{#3}
}

% param 1 = name of connector 1 (must exist)
% param 2 = name of connector 2 (must exist)
% param 3 = label to be used on linear unit
\newcommand{\RWOverlapReverseAbove}[3] {
  \draw (#1) -- (#2);
 %% \coordinate (#1) at #2;
  \draw ($(#1) +(0.1,-\RWConnectorHalfHeight)$) -- ($(#1) +(0,-\RWConnectorHalfHeight)$);
  \draw ($(#1) +(0.1,+\RWConnectorHalfHeight)$) -- ($(#1) +(0,+\RWConnectorHalfHeight)$);
  \RWLabelLinearUnitAbove{#1}{#2}{#3}
}

% param 1 = name of connector 1 (must exist)
% param 2 = name of connector 2 (must exist)
% param 3 = label to be used on linear unit
\newcommand{\RWOverlapBelow}[3] {
  \draw (#1) -- (#2);
 %% \coordinate (#1) at #2; 
  \draw ($(#2) -(0.1,-\RWConnectorHalfHeight)$) -- ($(#2) -(0,-\RWConnectorHalfHeight)$);
  \draw ($(#2) -(0.1,+\RWConnectorHalfHeight)$) -- ($(#2) -(0,+\RWConnectorHalfHeight)$);
  \RWLabelLinearUnitBelow{#1}{#2}{#3}
}

\newcommand{\RWOverlapReverseBelow}[3] {
  \draw (#1) -- (#2);
 %% \coordinate (#1) at #2;
  \draw ($(#1) +(0.1,-\RWConnectorHalfHeight)$) -- ($(#1) +(0,-\RWConnectorHalfHeight)$);
  \draw ($(#1) +(0.1,+\RWConnectorHalfHeight)$) -- ($(#1) +(0,+\RWConnectorHalfHeight)$);
  \RWLabelLinearUnitBelow{#1}{#2}{#3}
}


%%%%%%%%%%%%%%%%%%%%%%%%%%%%%%%%%%%%%%%%%%%%%%%%%%%%%%%%%%%%
%% Points
%%%%%%%%%%%%%%%%%%%%%%%%%%%%%%%%%%%%%%%%%%%%%%%%%%%%%%%%%%%%

%    /----
% --------
% param 1 = name for center of point (must not exist)
% param 2 = name for center of left connector (must not exist)
% param 3 = name for center of normal connector (must not exist)
% param 4 = name for center of reverse connector (must not exist)
% param 5 = coordinate for center of point
\newcommand{\RWPoint}[5]{
  \coordinate (#1) at #5;
  \RWConnector{#2}{($#5 + (-\RWPointHalfWidth,0)$)}
  \RWConnector{#3}{($#5 + (\RWPointHalfWidth,0)$)}
  \RWConnector{#4}{($#5 + (\RWPointHalfWidth,\RWPointHeight)$)}

  \draw (#2) -- #5;
  \draw #5 -- (#3);
  \draw #5 -- (#4);
}

% --------
%    \----

% param 1 = name for center of point (must not exist)
% param 2 = name for center of left connector (must not exist)
% param 3 = name for center of normal connector (must not exist)
% param 4 = name for center of reverse connector (must not exist)
% param 5 = coordinate for center of point
\newcommand{\RWPointUpsideDown}[5]{
  \coordinate (#1) at #5;
  \RWConnector{#2}{($#5 + (-\RWPointHalfWidth,0)$)}
  \RWConnector{#3}{($#5 + (\RWPointHalfWidth,0)$)}
  \RWConnector{#4}{($#5 + (\RWPointHalfWidth,-\RWPointHeight)$)}

  \draw (#2) -- #5;
  \draw #5 -- (#3);
  \draw #5 -- (#4);
}

% ----\
% ---------
% param 1 = name for center of point (must not exist)
% param 2 = name for center of right connector (must not exist)
% param 3 = name for center of normal connector (must not exist)
% param 4 = name for center of reverse connector (must not exist)
% param 5 = coordinate for center of point
\newcommand{\RWPointReverse}[5]{
  \coordinate (#1) at #5;
  \RWConnector{#2}{($#5 + (+\RWPointHalfWidth,0)$)}
  \RWConnector{#3}{($#5 + (-\RWPointHalfWidth,0)$)}
  \RWConnector{#4}{($#5 + (-\RWPointHalfWidth,\RWPointHeight)$)}

  \draw (#2) -- #5;
  \draw #5 -- (#3);
  \draw #5 -- (#4);
}


% ---------
%     /----
% param 1 = name for center of point (must not exist)
% param 2 = name for center of right connector (must not exist)
% param 3 = name for center of normal connector (must not exist)
% param 4 = name for center of reverse connector (must not exist)
% param 5 = coordinate for center of point
\newcommand{\RWPointReverseUpsideDown}[5]{
  \coordinate (#1) at #5;
  \RWConnector{#2}{($#5 + (+\RWPointHalfWidth,0)$)}
  \RWConnector{#3}{($#5 + (-\RWPointHalfWidth,0)$)}
  \RWConnector{#4}{($#5 + (-\RWPointHalfWidth,-\RWPointHeight)$)}

  \draw (#2) -- #5;
  \draw #5 -- (#3);
  \draw #5 -- (#4);
}



%%%%%%%%%%%%%%%%%%%%%%%%%%%%%%%%%%%%%%%%%%%%%%%%%%%%%%%%%%%%
%% LongPoints
%%%%%%%%%%%%%%%%%%%%%%%%%%%%%%%%%%%%%%%%%%%%%%%%%%%%%%%%%%%%

%    /----
% --------
% param 1 = name for center of point (must not exist)
% param 2 = name for center of left connector (must not exist)
% param 3 = name for center of normal connector (must not exist)
% param 4 = name for center of reverse connector (must not exist)
% param 5 = coordinate for center of point
\newcommand{\RWLongPoint}[5]{
  \coordinate (#1) at #5;
  \RWConnector{#2}{($#5 + (-3*\RWPointHalfWidth,0)$)}
  \RWConnector{#3}{($#5 + (\RWPointHalfWidth,0)$)}
  \RWConnector{#4}{($#5 + (\RWPointHalfWidth,\RWPointHeight)$)}

  \draw (#2) -- #5;
  \draw #5 -- (#3);
  \draw #5 -- (#4);
}

% --------
%    \----

% param 1 = name for center of point (must not exist)
% param 2 = name for center of left connector (must not exist)
% param 3 = name for center of normal connector (must not exist)
% param 4 = name for center of reverse connector (must not exist)
% param 5 = coordinate for center of point
\newcommand{\RWLongPointUpsideDown}[5]{
  \coordinate (#1) at #5;
  \RWConnector{#2}{($#5 + (-3*\RWPointHalfWidth,0)$)}
  \RWConnector{#3}{($#5 + (\RWPointHalfWidth,0)$)}
  \RWConnector{#4}{($#5 + (\RWPointHalfWidth,-\RWPointHeight)$)}

  \draw (#2) -- #5;
  \draw #5 -- (#3);
  \draw #5 -- (#4);
}

% ----\
% ---------
% param 1 = name for center of point (must not exist)
% param 2 = name for center of right connector (must not exist)
% param 3 = name for center of normal connector (must not exist)
% param 4 = name for center of reverse connector (must not exist)
% param 5 = coordinate for center of point
\newcommand{\RWLongPointReverse}[5]{
  \coordinate (#1) at #5;
  \RWConnector{#2}{($#5 + (\RWLongPointHalfWidth,0)$)}
  \RWConnector{#3}{($#5 + (-\RWPointHalfWidth,0)$)}
  \RWConnector{#4}{($#5 + (-\RWPointHalfWidth,\RWPointHeight)$)}

  \draw (#2) -- #5;
  \draw #5 -- (#3);
  \draw #5 -- (#4);
}


% ----\
% ---------
% param 1 = name for center of point (must not exist)
% param 2 = name for center of right connector (must not exist)
% param 3 = name for center of normal connector (must not exist)
% param 4 = name for center of reverse connector (must not exist)
% param 5 = coordinate for center of point
\newcommand{\RWSemiLongPointReverse}[5]{
  \coordinate (#1) at #5;
  \RWConnector{#2}{($#5 + (\RWPointHalfWidth,0)$)}
  \RWConnector{#3}{($#5 + (-\RWSemiLongPointHalfWidth,0)$)}
  \RWConnector{#4}{($#5 + (-\RWPointHalfWidth,\RWPointHeight)$)}

  \draw (#2) -- #5;
  \draw #5 -- (#3);
  \draw #5 -- (#4);
}




% ---------
%     /----
% param 1 = name for center of point (must not exist)
% param 2 = name for center of right connector (must not exist)
% param 3 = name for center of normal connector (must not exist)
% param 4 = name for center of reverse connector (must not exist)
% param 5 = coordinate for center of point
\newcommand{\RWLongPointReverseUpsideDown}[5]{
  \coordinate (#1) at #5;
  \RWConnector{#2}{($#5 + (\RWPointHalfWidth,0)$)}
  \RWConnector{#3}{($#5 + (-3*\RWPointHalfWidth,0)$)}
  \RWConnector{#4}{($#5 + (-\RWPointHalfWidth,-\RWPointHeight)$)}

  \draw (#2) -- #5;
  \draw #5 -- (#3);
  \draw #5 -- (#4);
}



%%%%%%%%%%%%%%%%%%%%%%%%%%%%%%%%%%%%%%%%%%%%%%%%%%%%%%%%%%%%
%% Junctions
%%%%%%%%%%%%%%%%%%%%%%%%%%%%%%%%%%%%%%%%%%%%%%%%%%%%%%%%%%%%

%    /----
% --------
% param 1 = name for center of point (must not exist)
% param 2 = name for center of left connector (must not exist)
% param 3 = name for center of normal connector (must not exist)
% param 4 = name for center of reverse connector (must not exist)
% param 5 = coordinate for center of point
% param 6 = label for linner unit 1 - left
% param 7 = label for linner unit 2 - normal
% param 8 = label for linner unit 3 - reverse
% param 9 = label for the point
\newcommand{\RWJunction}[9]{
  %% We assume RWTempA, RWTempB and RWTempC coordinate names are not
  %% used. These can be reused as they are just overwritten.
  \RWPoint{#1}{RWTempA}{RWTempB}{RWTempC}{#5} 
  \RWConnector{#2}{($(RWTempA) + (-\RWJunctionUnitLength,0)$)}
  \RWConnector{#3}{($(RWTempB) + (+\RWJunctionUnitLength,0)$)}
  \RWConnector{#4}{($(RWTempC) + (+\RWJunctionUnitLength,0)$)}
  \draw (RWTempA) -- (#2);
  \draw (RWTempB) -- (#3);
  \draw (RWTempC) -- (#4);
  \node[anchor = north] at ($(RWTempA)!.5!(#2)$) {#6};
  \node[anchor = north] at ($(RWTempB)!.5!(#3)$) {#7};
  \node[anchor = south] at ($(RWTempC)!.5!(#4)$) {#8};
  \node[anchor = north] at ($(RWTempA)!.5!(RWTempB)$) {#9};
}

% ----\
% ---------
% param 1 = name for center of point (must not exist)
% param 2 = name for center of right connector (must not exist)
% param 3 = name for center of normal connector (must not exist)
% param 4 = name for center of reverse connector (must not exist)
% param 5 = coordinate for center of point
% param 6 = label for linner unit 1 - right
% param 7 = label for linner unit 2 - normal
% param 8 = label for linner unit 3 - reverse
% param 9 = label for the point
\newcommand{\RWJunctionReverse}[9]{
  %% We assume RWTempA, RWTempB and RWTempC coordinate names are not
  %% used. These can be reused as they are just overwritten.
  \RWPointReverse{#1}{RWTempA}{RWTempB}{RWTempC}{#5}
  \RWConnector{#2}{($(RWTempA) + (+\RWJunctionUnitLength,0)$)}
  \RWConnector{#3}{($(RWTempB) + (-\RWJunctionUnitLength,0)$)}
  \RWConnector{#4}{($(RWTempC) + (-\RWJunctionUnitLength,0)$)}
  \draw (RWTempA) -- (#2);
  \draw (RWTempB) -- (#3);
  \draw (RWTempC) -- (#4);
  \node[anchor = north] at ($(RWTempA)!.5!(#2)$) {#6};
  \node[anchor = north] at ($(RWTempB)!.5!(#3)$) {#7};
  \node[anchor = south] at ($(RWTempC)!.5!(#4)$) {#8};
  \node[anchor = north] at ($(RWTempA)!.5!(RWTempB)$) {#9};
}

%%%%%%%%%%%%%%%%%%%%%%%%%%%%%%%%%%%%%%%%%%%%%%%%%%%%%%%%%%%%
%% Platforms
%%%%%%%%%%%%%%%%%%%%%%%%%%%%%%%%%%%%%%%%%%%%%%%%%%%%%%%%%%%%

% Make a platform between two connectors assuming the connectors have
% the same y values and that the first is positioned to the left of
% the second.
% A platform also joins the connectors with a track (a line)
% param 1 = name of the platform (must not exist)
% param 2 = name of the first connector (must exist)
% param 3 = name of the second connector (must exist)
\newcommand{\RWPlatformAbove}[3]{
  \draw[fill=gray] ($(#2) + (\RWPlatformGapFromConnector,\RWPlatformGapFromTrack)$) -- ($(#3) + (-\RWPlatformGapFromConnector,\RWPlatformGapFromTrack)$) -- 
    ($(#3) + (-\RWPlatformGapFromConnector,\RWPlatformGapFromTrack) + (0,\RWPlatformHeight)$) -- ($(#2) + (\RWPlatformGapFromConnector,\RWPlatformGapFromTrack) + (0,\RWPlatformHeight)$) --
    cycle;

  \coordinate (#1) at ($(#2)!.5!(#3) + (0,\RWPlatformGapFromTrack) + 0.5*(0, \RWPlatformHeight)$);
  \draw (#2) -- (#3);
}

% Make a platform between two connectors assuming the connectors have
% the same y values and that the first is positioned to the left of
% the second.
% A platform also joins the connectors with a track (a line)
% param 1 = name of the platform (must not exist)
% param 2 = name of the first connector (must exist)
% param 3 = name of the second connector (must exist)
\newcommand{\RWPlatformBelow}[3]{
  \draw[fill=gray] ($(#2) + (\RWPlatformGapFromConnector,-\RWPlatformGapFromTrack)$) -- ($(#3) + (-\RWPlatformGapFromConnector,-\RWPlatformGapFromTrack)$) -- 
    ($(#3) + (-\RWPlatformGapFromConnector,-\RWPlatformGapFromTrack) + (0,-\RWPlatformHeight)$) -- ($(#2) + (\RWPlatformGapFromConnector,-\RWPlatformGapFromTrack) + (0,-\RWPlatformHeight)$) --
    cycle;

  \coordinate (#1) at ($(#2)!.5!(#3) + (0,-\RWPlatformGapFromTrack) + 0.5*(0, -\RWPlatformHeight)$);
  \draw (#2) -- (#3);
}

% param 1 = name of the platform to label (must exist)
% param 2 = label to be used on platform
\newcommand{\RWLabelPlatformAbove}[2]{
  \node [anchor = south] at ($(#1) + 0.5*(0, \RWPlatformHeight)$) {#2};
}

% param 1 = name of the platform to label (must exist)
% param 2 = label to be used on platform
\newcommand{\RWLabelPlatformBelow}[2]{
  \node [anchor = north] at ($(#1) + 0.5*(0, -\RWPlatformHeight)$) {#2};
}



\newcommand{\RWTrainAbove}[3]{
  \draw[fill=gray] ($(#2) + (2.5*\RWPlatformGapFromConnector,1.2*\RWPlatformGapFromTrack)$) -- ($(#3) + (-2.5*+\RWPlatformGapFromConnector,1.2*\RWPlatformGapFromTrack)$) -- 
    ($(#3) + (-2.7*\RWPlatformGapFromConnector,1.2*\RWPlatformGapFromTrack) + (0,\RWPlatformHeight)$) -- ($(#2) + (2.7*\RWPlatformGapFromConnector,1.2*\RWPlatformGapFromTrack) + (0,\RWPlatformHeight)$) --
    cycle;

  \coordinate (#1) at ($(#2)!.5!(#3) + (0,\RWPlatformGapFromTrack) + 0.5*(0, \RWPlatformHeight)$);
  \draw (#2) -- (#3);
}


% param 1 = name of connector (must not exist)
% param 2 = coordinate for center of connector
\newcommand{\RWSignal}[3]{
  \coordinate (#1) at #2;
  \draw ($(#1) + (-0.3,0)$) -- ($(#1) +(-0.3,+\RWSignalHeight)$); 
  
  \node[draw, inner sep=0pt, minimum size= 7pt] (A) at ($(#1) +(0,+\RWSignalHeight)$) {};

  \draw  ($(#1) +(-0.3,+\RWSignalHeight)$) -- (A.west);
  \node at ($(#1) +(0.6,+\RWSignalHeight)$)  {#3};
}

\newcommand{\RWReverseSignal}[3]{
  \coordinate (#1) at #2;
  \draw ($(#1) + (0.3,0)$) -- ($(#1) +(0.3,+\RWSignalHeight)$); 
  
  \node[circle, draw, inner sep=0pt, minimum size= 7pt] (A) at ($(#1) +(0,+\RWSignalHeight)$) {};
  \node at  ($(#1) +(0,1.7*\RWSignalHeight)$) {#3};
  \draw  (A.east) --  ($(#1) +(0.3,+\RWSignalHeight)$);
}
