\section{Introduction}

% “Introduction: in general, could you introduce the problem in a
% sentence or two for the non-expert? What is an interlocking”


The European Rail Traffic Management System (ERTMS) / European Train
Control System (ETCS) is a European signalling, control and train
protection system designed to allow for high speed travel, to increase
capacity, and to facilitate cross-border traffic movements
\cite{ERTMS}. ERTMS/ETCS is a complex system of systems, made up by
distributed components. It is specified at four different levels,
where each level defines a different use as a train control system. In
our paper we consider ERTMS/ETCS Level 2, which is characterised by
continuous communications between trains and a radio block centre.

The switch from classical railway signalling systems to ERTMS/ETCS
train control poses a number of research questions for the formal methods
community. Can safety be guaranteed? Can formal methods be used to
confirm that such a switch improves capacity? Is it possible to
predict capacity using formal methods? To address such questions it is
necessary to develop and analyse timed or hybrid models. ERTMS/ETCS
Level 2 takes speed and braking curves of each individual train into
account. These determine the train's braking point well in advance of
the end of authority that the signalling system had granted to this
train. Such an approach is in contrast to classical signalling
systems, which treat all trains in the same way. Therefore, they need
to be designed for worst case braking. Consequently, in
formal safety analysis, such traditional systems can be treated on a purely
logical level, ignoring the aspect of time -- see, e.g.,
\cite{sttt14,JamesMNRST14}.

An ERTMS/ETCS system consists of a controller, an interlocking (a
specialised computer that determines if a request from the controller
is ``safe''), a radio block centre, track equipment, and a number of
trains. While the ERTMS/ ETCS standard details the interactions between
trains and track equipment (e.g., in order to obtain concise train
position information) and radio block centre and trains (e.g., to hand
out movement authorities), the details of how controller, interlocking
and radio block centre interact with each other are left to the
suppliers of signalling solutions such as our industrial partner
Siemens Rail Automation UK. In this paper we work with
the implementation as realised by Siemens. In the following we refer
to this system simply as ERTMS.

One development step when building an ERTMS system consists of
developing a so-called detailed design. Given geographical data such
as a specific track layout and what routes through this track layout
shall be used, the detailed design adds a number of tables that
determine the location specific behaviour of interlocking and radio
block centre. The objective of our modelling is to provide a formal
argument that a given detailed design is safe. Here we focus on
collision freedom, though our model is extensible for dealing with
further safety properties, and possibly also with performance
analysis.

We base our modelling approach on Real-Time Maude, which is a language
and tool supporting the formal object-oriented specification and
analysis of real-time and hybrid systems. In order to obtain a faithful model of
ERTMS/ETCS level 2 on the design level, we follow a systematic
approach, established by the Swansea Railway Verification
Group.
%% %
%% First, we develop a domain specific language that allows us to
%% represent the rail yard under discussion.
%% %
%% Then, we systematically identify the system's entities and determine the
%% information flow between them.
%% %
%% Finally, we provide a number of translation tables: static elements
%% such as the track plan and the various tables are specified as data
%% types using Maude equations; the ERTMS/ETCS components are represented
%% as objects; their message exchange according to the rules laid down
%% in the standard or by Siemens is captured by rewrite rules.

This paper extends our location-specific modelling presented in past work~\cite{old} to a generic and far more detailed modelling. It is
organised as follows. First, we introduce the ERTMS Level 2 standard,
%describe the detailed design of a pass-through station, 
and briefly discuss high level safety properties for ERTMS. Then, we
give a short presentation of Real-Time Maude with a focus on standard
specification techniques for hybrid systems. In Section 4, we present
our modelling of ERTMS in Real-Time Maude, discussing each component
in detail. In Section 5, we validate our model by simulation and
error injection. Finally, we present model checking results and put
our approach in the context of related work.

