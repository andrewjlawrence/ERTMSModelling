\section{Modelling ERTMS in Maude}

To the best of our knowledge, our modelling of ERTMS is the first one
comprising all ERTMS subsystems required for the control cycle in
ERTMS/ETCS Application Level 2, c.f.\ Figure 6 in the ERTMS/ETCS
System Requirements Specification \cite{spec}. For simplicity, we
consider only uni-directional rail yards, as these exhibit many of the
components of bi-directional rail yards, but are of a lower complexity
with regards to the number of routes required within the model. Also,
we make the standard assumption that trains have no length. This is
the typical abstraction when one deals with trains whose length is
shorter than any track length in the given scheme plan. For a detailed
discussion of the topic see, e.g., our publication discussing train
length \cite{JamesMNRST14}.

In the following, we provide an overview of our model:\footnote{The models are available at:\\
{\tt \footnotesize
 http://www.cs.swan.ac.uk/\%7Ecsmarkus/ProcessesAndData/Models}} first we discuss the static
data types; then we look at the instantaneously reacting
sub-systems, i.e., controller, interlocking, and RBC; next, we
describe how we capture  train behaviour, which requires
differential equations describing motion; finally, we address how to
express collision-freedom. We note that our model is generic, with only
location specific data as a parameter. This location specific data has been encoded manually, however this process could be automated within OnTrack~\cite{james14c}.

%% {\sl For the purpose of reviewing, our models are available at
%% \begin{center}
%% \url{http://www.cs.swan.ac.uk/%7Ecsmarkus/Station.zip}.
%% \end{center}
%% The \%7E stands for $\sim$ -- in case the link is not clickable. In
%% case of acceptance, we will make our code available to the scientific
%% community via a stable webpage.}

\subsection{Datatypes: Location Specific Data and Messages}
%%%Topology and control/rbc tables.
%Utilising our domain specific language, see, e.g., \cite{james14}, 
We model the rail topology as a connected collection of tracks,
points, and routes and provide a systematic translation into
Maude. For the example given in Figure \ref{fig:station}, the location
specific data Maude is encoded as follows:
%
\begin{lstlisting}[columns=fixed]
sort RouteName . ops RouteName1A ... : -> RouteName .
sort Track .     ops AA AB AC ... : -> Track        .
sort Point .     ops P1 P2 : -> Point               . 
\end{lstlisting}
%
The connection between tracks is given by a \verb|next| function. If
the track under discussion is a point, as, e.g., track \verb|AB|, it
has two potential successors, namely \verb|AC| and
\verb|BC|, depending on the current setting of the point.
%
\begin{lstlisting}[columns=fixed]
op next : Track PointPos -> Track . var PPos : PointPos .
eq next(AA,PPos) = AB                                   .
eq next(AB,normal) = AC . eq next(AB,reverse) = BC      .
\end{lstlisting}

The various tables (clear and release tables for the scheme plan, the
tables of the RBC) are encoded by defining a function for each
column. A typical example is the ``Clear Tracks''
column\footnote{Compared to the given control table, we add {\tt
RouteName4} to cover the exit track.} of the control table in
Figure \ref{fig:station}:
%
\begin{lstlisting}[columns=fixed]
op clearTracks : RouteName -> SetOfTracks  .
eq clearTracks(RouteName1A) = (AA, AB, AC) . 
...
eq clearTracks(RouteName4)  = empty        .
\end{lstlisting}

The ERTMS components exchange a number of messages, see Figure
\ref{fig:arch}. As we are dealing with a single geographic region,
controller, interlocking, and RBC are unique. Thus, for most messages
no object identifier is needed:
%
\begin{lstlisting}[columns=fixed]
msgs routerequest, proceedrequest, ... : RouteName -> Msg .
\end{lstlisting}
This is in contrast to messages involving trains. For instance, the message 
\begin{lstlisting}[columns=fixed]
msg magrant : Oid Nat -> Msg .
\end{lstlisting}
grants a movement authority (encoded as a natural number, determining
the position to which the train is allowed to travel) to a specific
train with an object identifier of type \verb|Oid|.
%
Messages are urgent, i.e., their processing time is 0:
\begin{lstlisting}[columns=fixed]
eq mte(M:Msg) = 0 .
\end{lstlisting}


\subsection{Instantaneously Reacting Sub-Systems}
\label{ssec:instan}

The processing time of controller, interlocking, and RBC is negligible
compared to the time that it takes a train to pass a track. Thus, in
our modelling we assume that these three components react
instantaneously. In Maude this is expressed by saying that these
components do not pose any time constrains. Here, written for the
controller:
%
\begin{lstlisting}[columns=fixed]
eq mte(< O1 : Controller | >) = INF .
\end{lstlisting} 

\vspace{1ex}
\noindent
{\bf Controller.}
An ERTMS controller issues route requests. For a general safety
analysis, a \emph{random controller} that can make any order of route
requests should be considered:
%
\begin{lstlisting}[columns=fixed]
op randomRoute : -> RouteName .
rl randomRoute => RouteName1A . 
... 
rl randomRoute => RouteName4  .
\end{lstlisting}
%
However, it is also possible to perform safety analysis relatively to
a specific strategy, e.g., a \emph{round-robin controller} that
requests routes as follows -- 1A first, followed by 1B, until route
4, starting over with 1A again:
\begin{lstlisting}[columns=fixed]
eq routeOrder = (RouteName1A : RouteName1B : ... : RouteName4) .
\end{lstlisting}
%
Yet another parameter are the times at which the controller makes
route requests. For both controllers we work with a constant
frequency.
 

\vspace{1ex}
\noindent
{\bf Interlocking.}
In rail control systems, the interlocking provides a safety layer
between controller and track.
%
%%%In ERMTS it further informs the RBC which routes can be granted. 
%
To this end, it monitors the physical
rail yard (\verb|occ| says which tracks are currently occupied,
\verb|pointPositions| says for each point if it is in normal or in
reverse position), manages locks (\verb|pointslocked| says if a point
is currently locked by a route), and stores which routes are
currently set (\verb|routeset|):
%
\begin{lstlisting}[columns=fixed]
class Inter |  routeset : MapRouteName2Bool, 
               pointslocked : MapPoint2Bool,
               occ : MapTrack2Bool, 
               pointPositions : MapPoint2PointPos  .
\end{lstlisting}

The interlocking is a passive component, i.e., only upon receiving a
message it possibly changes its state and/or sends a message. A
typical rule for preserving safety is the following:
%
\begin{lstlisting}[columns=fixed]
crl  routerequest(RN1) 
     < O : Inter |  routeset : MAPRNB1, 
                    occ : MAPTB1, pointslocked : MAPPB3 >
   => < O : Inter | > if (not checkClear(RN1, MAPTB1)) or 
                         pointsLocked(RN1, MAPPB3) .
\end{lstlisting}
%
A route request by the controller is ignored in case that the tracks
specified in the clear table for route \verb|RN1| are occupied or the
points of route \verb|RN1| are locked in different positions. 
 

\vspace{1ex}
\noindent
{\bf RBC.}
The RBC mediates between requests from the trains to extend their
movement authorities and the successful route requests by the
controller. To this end it reconciles two different views on the rail
yard: trains use continuous data to represent their position (in our
model the distance from the leftmost point of the rail yard); the
interlocking uses discrete data (track occupation, set routes, point
positions) in its logic.
%
In our model, we take a rather simplified and also abstract view on
the challenges involved. We make the assumption that trains request a
new movement authority only on the track on which their current
authority ends. Furthermore, we abstract the mapping between
continuous and discrete data to the two tables presented in Figure
\ref{fig:rbctables}.

In our model, the RBC only holds information on successful route
requests (in \verb|availableRoutes|) and for which trains
(characterised by their \verb|Oid|) it currently has an open ``request
to proceed'' (in \verb|designatedRoutes|):
%
\begin{lstlisting}[columns=fixed]
class RBC | availableRoutes  : SetOfRouteNames, 
            designatedRoutes : MapOid2RouteName  .
\end{lstlisting}

Also, the RBC is a passive system component. A typical reaction is the
following: When the interlocking sends a ``proceed message'' for a
route \verb|RN|, the RBC sends a new ``end of authority'' to the train
and removes the corresponding request from its internal state.
%
\begin{lstlisting}[columns=fixed]
eq   proceedgrant(RN) < O2 : RBC | designatedRoutes : TRN > 
   = magrant(getTrain(RN, TRN), endOfAuthority(RN)) 
     < O2 : RBC | designatedRoutes : removeRoute(_,_) >  . 
\end{lstlisting}


%%%Time Dependant Reacting Sub-Systems
\subsection{Trains}
The \verb|Train| class is the only time dependent entity in our
model. It is designed as an automaton with four
states \verb|stop|, \verb|acc| for accelerating, \verb|cons| for
constant speed, and \verb|brake|. There are transitions \verb|stop|
$\to $ \verb|acc| $\to $ \verb|cons| $\to $ \verb|brake|,
and \verb|acc| $\to$ \verb|brake| and vice versa. In addition, it has
fields representing the current distance (relative to a given
reference point 0), speed, acceleration, movement authority (relative
to 0), maximum speed, and the current track segment.
\begin{lstlisting}
class Train | state : TrainState, dist : NNegRat,
              speed : NNegRat, ac : NNegRat, ma : NNegRat,
              tseg : Track, maxspeed : NNegRat .
\end{lstlisting}
We assume that acceleration is linear, and -- apart from Scenario 3 in Section~\ref{sec:errorInjection} -- use a value of 1 for both acceleration and deceleration. Trains move according to Newton's laws, i.e., if at time $0$ a train is at \verb|DT| with speed \verb|S| and acceleration \verb|A|, then the speed at time \verb|R| is \verb|S + A * R| and the location is \verb|DT + S * R + A * R * R / 2|. Its braking distance \verb|bd(S,A)| is \verb|S * S / 2 * A|. In the following we show the rule for a train in the accelerating state.

\begin{lstlisting}
crl [acc] :
 < O1 : Inter | pointPositions : PointSettings >
 delta(< O : Train | state : acc, dist : DT, speed : S,
         ac : A, ma : MA, tseg : AN, maxspeed : MAX >, R)
      =>
      < O1 : Inter | pointPositions : PointSettings >
      trackseg(PointSettings, < O : Train |  
      state : if (S + A * R == MAX)     
              then cons
	      else (if R == mteMA(DT,S,A,MA)
	            then brake
	            else acc fi) fi, 
      dist : DT + S * R + R * R * A * (1/2),
      speed : S + A * R > ) if not AN == Exit .
\end{lstlisting}
The rule takes as argument a train object and a time \texttt{R} and computes its new configuration after time \texttt{R}. It is sufficient to list those attributes that are updated, here \texttt{speed}, location, and, possibly, its state.
The operator \verb|trackseg|  takes the new
location of the train and the \verb|PointSettings| from the
interlocking and returns a new train object.
In the case that the train has entered a new track it will
update the train object accordingly.

The time \verb|R| is determined by the maximal time elapse which is,
in the acceleration state, the minimum of the following three
cases. 1) maximum speed is reached, 2) the end of a track segment is
reached, 3) the distance to the movement authority is not greater than
the required braking distance.
\begin{lstlisting}
ceq mte (< O : Train | state : acc,  dist : DT, speed : S, 
           ac : A, ma : MA , tseg : AN, maxspeed : MAX >)
           = min((MAX monus S) / A,
                ((endof(AN) + 1) monus DT) / S,
                mteMA(DT,S,A,MA))  if S > 0 .
eq mteMA(DT, S, A, MA) = (((MA monus 1) monus DT) monus 
                         (S * S / (2 * A))) / ( 2 * S) .
\end{lstlisting}
In case 1) we used \texttt{monos} for the maximum of the difference between two numbers and \texttt{0}. 
For cases 2) and 3) the calculation of \texttt{mte} involves quadratic
equations.  From \texttt{DT + S * R + A * R * R / 2 < endof(AN) + 1} we
could determine \texttt{R} using an approximation via Newton's
method. However, since \texttt{0 < A * R * R / 2 <= A * R / 2 < 1}, our
assumption \texttt{A = 1} and as \texttt{0 < R <= 1} holds thanks to the default
tick, we approximate the quadratic term either from below or from
above depending on the context: in the case of entering a new track we
ignore the quadratic term as we want to be on the new track
already; in the case of calculating where to start braking, we bring
the event slightly forward, i.e., instead of braking exactly, we start
slightly too early. Both approximations are justified by the
default tick. Finally, to keep the size of rational numbers under
control, we determine the sampling time point with a precision of
\texttt{1/1000}.


%Also note that each train system will build in error
%margins in case of braking, which are larger than the ones used here,
%and never brake exactly to the point (Cf also section on validation).
%Uli's formulation:
%An important question is whether our modelling is complete, that is, whether
%it is able to detect any possible error (???). The criteria for completeness
%given in [2] appear to be satisfied by our modelling except for the conditions
%... which are only approximately satisfied due to rounding in connection with
%the square root function. We conjecture that an appropriate weakening of
%the conditions ... still suffices to imply completeness, but leave this 
%to further work. The property of being non-zeno, required in [2], can be 
%seen for our system as follows. ...





\subsection{Safety Condition}
\label{sec:safetycondmodelling}
For classical railway signalling, we established the following
finitisation theorem: if a signalling system is collision free for two
trains, then it is collision free for any number of
trains \cite{sttt14}. We conjecture that this result carries over to
ERTMS and consider our ERTMS system to be safe if -- within the scheme
plan under consideration -- two trains are always more than, say, 40m
apart. Thus, we check for the invariant ``no collisions'':
%
\begin{lstlisting}[columns=fixed]
   eq { REST  < train1 : Train |  tseg : T1 , dist : N1 >
              < train2 : Train |  tseg : T2 , dist : N2 > }
      |= nocrashDistance(train1, train2) 
   =
      ( ( not (T1 == Entry) and not (T2 == Entry) and 
          not (T1 == Exit ) and not (T2 == Exit ) ) 
      and ( T1 == T2 or 
            T1 == next(T2, normal) or T1 == next(T2, reverse) or
            T2 == next(T1, normal) or T2 == next(T1, reverse) ) )
      implies ((N2 monus N1 > 100) or (N1 monus N2 > 100)) .
\end{lstlisting}
%
I.e., a configuration with two objects \verb|train1| and \verb|train2| of type
 train models the parameterised formula \verb|nocrashDistance| iff the
 state of the two trains objects under consideration are in the
 relation specified after the equal sign. Here, \verb|T1|
 and \verb|T2| are the tracks and \verb|N1| and
\verb|N2| are the positions on which the two trains are respectively.
In the formula we check that the trains are more than 100m apart,
provided 
%
they are not on the \verb|Entry| or \verb|Exit| track, and
%
provided they are on the same (\verb|T1 == T2|) or on adjacent tracks.

The second condition is necessary as we model positions from a single
reference point on the \verb|Entry| track. For instance, on the track
plan shown in Figure~\ref{fig:station}, we can have one train on track
\verb|BC| and another train on track \verb|AC|, both with the same
distance, though by no means colliding with each other.
%
We note that we use the value of 100m for our invariant. This is different
from the desired 40m, but necessary due to our time sampling
strategy: we sample the system only once every
second. Within this time, the distance between two trains can reduce
by maximally 60m as we consider trains that travel at a
maximum of 60m/s.

%%%%%%% As a suggestion - to see how it looks -- put in a separate subsection.
\subsection{Completeness}
An important question is whether our modelling is complete, that is
all errors can be detected by our
modelling. \"Olveczky and Meseguer give criteria for
completeness in object oriented Real-Time Maude~\cite{PO06}. Essentially, one
needs to prove that the maximal time elapse function is time
robust. This is clearly the case if we consider movement without
acceleration. It is almost all the time the case for our modelling
with acceleration, however the small shifts of the sampling points
require further analysis. We expect that a weakening of Theorem 4~\cite{PO06}, which takes approximation into account, holds.  A
necessary premise for this theorem is non-zenoness for which we give
the following argument.

Our modelling is non-zeno in the sense of Henzinger \cite{Henzinger2000} as
there are no cycles in the behaviour of the automaton which allow time
to converge.  The argument is that any cycle will involve the
accelerating state, which requires a new movement authority to be
granted that will extend the current movement authority by at least
one. This causes a minimal time elapse bounded away from zero by a
fixed amount since the speed of a train is limited.





