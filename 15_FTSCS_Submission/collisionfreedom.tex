For classical railway signalling, we established the following
finitisation theorem: if a signalling system is collision free for two
trains, then it is collision free for any number of
trains \cite{sttt14}. We conjecture that this result carries over to
ERTMS and consider our ERTMS system to be safe if -- within the scheme
plan under consideration -- two trains are always more than, say, 40m
apart. Thus, we check for the invariant ``no collisions'':
%
\begin{lstlisting}[columns=fixed]
   eq { REST  < train1 : Train |  tseg : T1 , dist : N1 >
              < train2 : Train |  tseg : T2 , dist : N2 > }
      |= nocrashDistance(train1, train2) 
   =
      ( ( not (T1 == Entry) and not (T2 == Entry) and 
          not (T1 == Exit ) and not (T2 == Exit ) ) 
      and ( T1 == T2 or 
            T1 == next(T2, normal) or T1 == next(T2, reverse) or
            T2 == next(T1, normal) or T2 == next(T1, reverse) ) )
      implies ((N2 monus N1 > 100) or (N1 monus N2 > 100)) .
\end{lstlisting}
%
I.e., a configuration with two objects \verb|train1| and \verb|train2| of type
 train models the parameterised formula \verb|nocrashDistance| iff the
 state of the two trains objects under consideration are in the
 relation specified after the equal sign. Here, \verb|T1|
 and \verb|T2| are the tracks and \verb|N1| and
\verb|N2| are the positions on which the two trains are respectively.
In the formula we check that the trains are more than 100m apart,
provided 
%
they are not on the \verb|Entry| or \verb|Exit| track, and
%
provided they are on the same (\verb|T1 == T2|) or on adjacent tracks.

The second condition is necessary as we model positions from a single
reference point on the \verb|Entry| track. For instance, on the track
plan shown in Figure~\ref{fig:station}, we can have one train on track
\verb|BC| and another train on track \verb|AC|, both with the same
distance, though by no means colliding with each other.
%
We note that we use the value of 100m for our invariant. This is different
from the desired 40m, but necessary due to our time sampling
strategy: we sample the system only once every
second. Within this time, the distance between two trains can reduce
by maximally 60m as we consider trains that travel at a
maximum of 60m/s.

%%%%%%% As a suggestion - to see how it looks -- put in a separate subsection.
\subsection{Completeness}
An important question is whether our modelling is complete, that is
all errors can be detected by our
modelling. \"Olveczky and Meseguer give criteria for
completeness in object oriented Real-Time Maude~\cite{PO06}. Essentially, one
needs to prove that the maximal time elapse function is time
robust. This is clearly the case if we consider movement without
acceleration. It is almost all the time the case for our modelling
with acceleration, however the small shifts of the sampling points
require further analysis. We expect that a weakening of Theorem 4~\cite{PO06}, which takes approximation into account, holds.  A
necessary premise for this theorem is non-zenoness for which we give
the following argument.

Our modelling is non-zeno in the sense of Henzinger \cite{Henzinger2000} as
there are no cycles in the behaviour of the automaton which allow time
to converge.  The argument is that any cycle will involve the
accelerating state, which requires a new movement authority to be
granted that will extend the current movement authority by at least
one. This causes a minimal time elapse bounded away from zero by a
fixed amount since the speed of a train is limited.


