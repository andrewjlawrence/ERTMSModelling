In rail control systems, the interlocking provides a safety layer
between controller and track.
%
%%%In ERMTS it further informs the RBC which routes can be granted. 
%
To this end, it monitors the physical
rail yard (\verb|occ| says which tracks are currently occupied,
\verb|pointPositions| says for each point if it is in normal or in
reverse position), manages locks (\verb|pointslocked| says if a point
is currently locked by a route), and stores which routes are
currently set (\verb|routeset|):
%
\begin{lstlisting}[columns=fixed]
class Inter |  routeset : MapRouteName2Bool, 
               pointslocked : MapPoint2Bool,
               occ : MapTrack2Bool, 
               pointPositions : MapPoint2PointPos  .
\end{lstlisting}

The interlocking is a passive component, i.e., only upon receiving a
message it possibly changes its state and/or sends a message. A
typical rule for preserving safety is the following:
%
\begin{lstlisting}[columns=fixed]
crl  routerequest(RN1) 
     < O : Inter |  routeset : MAPRNB1, 
                    occ : MAPTB1, pointslocked : MAPPB3 >
   => < O : Inter | > if (not checkClear(RN1, MAPTB1)) or 
                         pointsLocked(RN1, MAPPB3) .
\end{lstlisting}
%
A route request by the controller is ignored in case that the tracks
specified in the clear table for route \verb|RN1| are occupied or the
points of route \verb|RN1| are locked in different positions. 
