\section{Related Work}
%%% Todo: Markus
ERTMS is a complex system of systems, made up of distributed
components interconnected through standard (e.g.\ Euroradio) and
proprietary (e.g.\ Siemens-specific) protocols and algorithms. Our
approach reflects this by covering the full control cycle between
controller, interlocking, radio-block centre and trains. Our objective
is to verify the location specific data of railway designs in their
early development stages, accompanying a standard design process
performed by signalling companies such as our industrial partner
Siemens.

%% This is in
%% contrast to a number of other works, which focus on parts of ERTMS
%% only. 

%% Another
%% dimension is the question if the verification takes place on design or
%% implementation level.

Our approach to cover all components is different from several
verification approaches with a focus on a single component only.
%
Vu et al.~\cite{vu15} provide a generic and re-configurable model of
ERTMS Level 2 on the design level sharing our objective. They present
their model as a Kripke structure and verify high-level safety
properties such as head-to-head collision or derailment on a point.
%
%% They introduce a concept of virtual signals and
%% argue that this allows them to handle the assignment of movement
%% authorities in a way similar to the situation where conventional
%% signals are used. Their verification technology is SMT solving as
%% implemented in the RT-Tester tool-box. It is able to handle large
%% Danish rail stations.
Their approach
%
%% This reduction
abstracts from trains and the RBC and presumes these components to be
correctly implemented. Thus, their verification focuses on the
interlocking component.
%
%% %
%% On the modelling side, differences appear to be mostly due to national
%% pecularities. They model the Danish interlocking system, which is
%% based upon ``interlocking tables'' similar to our control
%% tables. Furthermore, they apply the technique of sequential
%% release. As we deal with uni-directional railways only, in our setting
%% release tables for points are sufficient.
%
Cimatti et al.~\cite{cimatti12} apply software model checking to
verify the implementation level of a subsystem responsible for the
allocation of logical routes to trains. The software under
consideration has been developed by Ansaldo-STS and is part of this
company's implementation of ERTMS Level 2. %% An example of a property
%% under consideration is ``no two different trains occupy the same
%% track''.  Cimatti et al.\ represent the software in the VELOS
%% specification language (resembling the C++ language), the properties
%% of interest in temporal logic, and discuss in detail the performance
%% of different model checkers. Thus,
They focus on software verification of a sub-component rather than on
location specific data for the whole system.
%
Nardone et al.~\cite{nardone14} develop a new, rail specific
specification language DSTM4Rail, an extension of hierarchical state
machines. They employ DSTM4Rail to the modelling of specific
functionalities of the ERTMS Radio Block Centre. Overall the objective
is to obtain a formal model of ERTMS requirements for system testing
purposes. %% The long term goal is integration into a model-driven
%% development processes.
This work is specialised to quality assurance for one ERTMS component.

The openETCS initiative~\cite{openETCS} sets out to provide
specifications that can be used for software generation for ETCS train
control components, track elements, and functionality to be integrated
in track side interlocking systems. This software development follows
a model-driven approach, where the methods and tools shall comply with
a SIL 4 development process. 

Chiappini et al.~\cite{chiappini10} work towards the
formalisation and validation of the overall ERTMS/ETCS
specifications. To this end, they formalise a reference subset
(including Movement Authority Management and RBC/RBC Handover) of the
system requirements through a set of concepts and diagrams in UML, and
through additional constraints in a defined controlled natural
language. This formalisation then undergoes an automatic validation
check covering questions concerning consistency, scenario
compatibility, and if certain properties hold. Their work puts the
ERTMS/ETCS specifications themselves under scrutiny. 
%%Their methods are semi-formal.




