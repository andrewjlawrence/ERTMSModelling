\section{Validation Through Simulation and Error Injection}
\label{sec:validation}
 Here we give a number of scenarios to illustrate that our modelling
 is able to capture typical errors that are made when designing ERTMS
 subsystems. Concerning verification tools, we rely on the model
 checking capabilities of the Real-Time Maude
 Tool~\cite{olveczky2008real} to provide the relevant
 counter-examples. In carrying out the verification, our starting
 point is that the generic models of the interlocking, RBC and trains
 are correct. However, we make no assumptions about the correctness of
 the instantiation of our modelling with concrete \emph{Control
   Tables}, \emph{Release Tables} and \emph{RBC tables}.

\subsection{Simulation}
We first demonstrate the behaviour of one train moving through the
rail yard in Figure~\ref{fig:station} with a start position on track
\texttt{AA} and a movement authority of 1498. For this we use the
Real Time Maude \texttt{trew} command to execute our model up to a
given time bound.
\begin{lstlisting}
(trew { 
  < inter1 : Inter | pointPositions : (P1 |-> normal,
                                       P2 |-> normal) , ... >
  < train1 : Train | state : acc, dist : 2, speed : 0, ac : 1, 
                     ma : 1498, tseg : AA , maxspeed : 60 > }
in time <= 39 .)
\end{lstlisting}
The train accelerates until it begins to brake at
the distance of 749.72m:
\begin{lstlisting}
Result ClockedSystem : { < inter1 : Inter | ...>
  < train1 : Train | ac : 1, dist : 1499446241/2000000,
    ma : 1498, maxspeed : 60, speed : 38671/1000, 
    state : brake, tseg : AA >} in time 38671/1000
\end{lstlisting}
A query one time step later shows that a movement authority request is made.
\begin{lstlisting}
{marequest(train1,AA) < inter1 : ...> 
 < train1 : Train | speed : 37671/1000, ... >} in time 39671/1000
\end{lstlisting}
Now, the system cannot progress, unless we add an RBC to
our configuration. 
%Now the system cannot progress, unless we add an (here trivial) rbc to
%our configuration. As no follow-up route is set the request is ignored.
\begin{lstlisting}
(trew { < inter1 : Inter | ... > < train1 : Train | ... >
 < rbc1 : RBC | availableRoutes : empty ,
                designatedRoutes : empty  >} in time <= 78 .)
\end{lstlisting}
As no follow-up route is available in the RBC, the train stops at 1497.46m.
\begin{lstlisting}
  {< inter1 : Inter | ... > < rbc1 : RBC | ... >
   < train1 : Train | dist : 1497446241/1000000, ma : 1498,
     speed : 0, state : stop, tseg : AA >} in time 38671/500
\end{lstlisting}
To continue, assume that we start in the configuration where the
interlocking has set \texttt{RouteName3} and the train has made a
movement authority request.
\begin{lstlisting}
(trew {marequest(train1,AA)
  < inter1 : Inter | routeset : RouteName3 |-> true,... >
  < train1 : Train | state : brake, dist : 760, speed : 37,
    ac : 1, ma : 1498, tseg : AA , maxspeed : 60 >
  < rbc1 : RBC | availableRoutes : (RouteName3), ...  >
  } in time <= 17 .)
\end{lstlisting}
Below we see that the authority is extended to 6499m, and P2 gets
locked. Time 17 is when the train crosses to track AB and can
accelerate to maximum speed.
\begin{lstlisting}
{ < inter1 : Inter | occ : (AA |-> false, AB |-> true),
  pointslocked : P2 |-> true, ... >
  < rbc1 : RBC | availableRoutes : empty, ... >
  < train1 : Train | dist : 3001/2, ma : 6499, speed : 52,
  state : acc,tseg : AB >} in time 17
\end{lstlisting}



%% As a second example we assume that we have two trains,
%% a slow \texttt{train1} at the end of \texttt{Route1A} and fast \texttt{train2} on \texttt{Entry}, both points settings are set to \texttt{normal}.
%% \begin{lstlisting}
%% (trew { < inter1 : Inter | ... >
%%     < train1 : Train | state : stop, dist : 3249, speed : 0,
%%     ac : 1,  ma : 3249, tseg : AC , maxspeed : 20 >
%%     < train2 : Train | state : stop, dist : 1, speed : 0,
%%     ac : 1,   ma : 1, tseg : Entry , maxspeed : 120 >
%%     < rbc1 : RBC | avaliableRoutes : empty , ... >
%%     < ctr1 : Controller | counter : 0, routes : routeOrder >
%% } in time <= 172 .)
%% \end{lstlisting}
%% The controler chooses routes in some order, first only \texttt{train1}
%% can get an authority and then \texttt{train2}, once \texttt{train1}
%% moved out of the way.
%% \begin{lstlisting}
%% ...
%% \end{lstlisting}
%% \texttt{Train2} comes to a stillstand on \verb|AC| at position
%% 3248.039 (just before the end of the granted \texttt{MA} at 3249.) 

\subsection{Error Injection}
\label{sec:errorInjection}
%%% Phil
We now show that our modelling is able to find errors in the design of various ERTMS components.  The following scenarios use our random
controller and check the safety condition presented in
Section~\ref{sec:safetycondmodelling}.  Furthermore, we model one slow train (max speed 20m/s) and one
fast train (max speed 60m/s).

\begin{lstlisting}
eq initState = {...
  < train1 : Train | state : stop, dist : 0, speed : 0,
    ac : 1,  ma : 1, tseg : Entry , maxspeed : 20 >
  < train2 : Train | state : stop, dist : 0, speed : 0,
    ac : 1,  ma : 1, tseg : Entry , maxspeed : 60 > ...} .
\end{lstlisting}

\vspace{1ex}
\noindent
{\bf Scenario 1 -- Incorrect Control Tables.} We
consider a scheme plan where the designer forgets to put track section
AC into the various interlocking tables in
Figure~\ref{fig:station}. Model checking highlights that two trains
may be within 100 meters of each other, with both trains on track AC.

\begin{lstlisting}
{...< train1 : Train | ac : 1, dist : 3249, ma: 3249,
      maxspeed : 20, speed : 0, state : stop, tseg : AC >
    < train2 : Train | ac : 1, dist : 1939979/625, ma : 6499,
      maxspeed : 60, speed : 60, state : cons, tseg : AC > ...}
\end{lstlisting}

\vspace{1ex}
\noindent
{\bf Scenario 2 -- Incorrect RBC Tables.} We
consider a scheme plan where the designer incorrectly calculates an EoA
of $3449$m for route 1A in the RBC tables given in
Figure~\ref{fig:rbctables}. Model checking highlights that two trains
may be within 100 meters with \verb|train1| overrunning onto track AD due
to the incorrect EoA and \verb|train2| approaching on AC.

\begin{lstlisting}
{...< train1 : Train | ac : 1,dist : 3449,ma : 3449,
      maxspeed : 20,speed : 0,state : stop,tseg : AD >
    < train2 : Train | ac : 1,dist : 12433788921/4000000,
      ma : 6499,maxspeed : 60, speed : 60,state : cons,
      tseg : AC > ...}
\end{lstlisting}

\vspace{1ex}
\noindent
{\bf Scenario 3 -- Incorrect Train Braking Parameters.}
The computation of the braking distance for a train is
based on various parameters, some of which may be incorrectly entered
by the driver. Hence the train's physical braking distance may differ
from the computed one. Below we consider a starting scenario where a
deceleration value of $1$ (hard-coded, for illustration) has been
incorrectly entered for \verb|train2|, whilst the physical train has a
deceleration value of $8/10$. The other train has correct parameters.
\begin{lstlisting}
{...< train1 : Train | state : stop, dist : 3249, speed : 0,
      ac : 1, ma : 6499, tseg : AD , maxspeed : 20 >
    < train2 : Train | state : stop, dist : 1, speed : 0,
      ac : 8/10, ma : 1, tseg : Entry , maxspeed : 60 > ...}
\end{lstlisting}
The incorrect parameter causes the two trains both to be on track
\verb|AF| within 100 meters of each other. This is due
to the incorrect behaviour of \verb|train2| which overruns its movement
authority thanks to its wrong braking parameter.
\begin{lstlisting}
{...< train1 : Train | ac : 1,dist : 15662341/2500,ma : 6499,
      maxspeed : 20,speed : 20,state : cons,tseg : AF >
    < train2 : Train | ac : 4/5,dist : 968593576867/156250000,
      ma : 7999,maxspeed : 60,speed : 60,state : cons,
      tseg: AF > ...}
  
\end{lstlisting}
