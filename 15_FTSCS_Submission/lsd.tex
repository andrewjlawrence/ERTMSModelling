%Utilising our domain specific language, see, e.g., \cite{james14}, 
We model the rail topology as a connected collection of tracks,
points, and routes and provide a systematic translation into
Maude. For the example given in Figure~\ref{fig:station}, the location
specific data Maude is encoded as follows:
%
\begin{lstlisting}[columns=fixed]
sort RouteName . ops RouteName1A ... : -> RouteName .
sort Track .     ops AA AB AC ... : -> Track        .
sort Point .     ops P1 P2 : -> Point               . 
\end{lstlisting}
%
The connection between tracks is given by a \verb|next| function. If
the track under discussion is a point, as, e.g., track \verb|AB|, it
has two potential successors, namely \verb|AC| and
\verb|BC|, depending on the current setting of the point.
%
\begin{lstlisting}[columns=fixed]
op next : Track PointPos -> Track . var PPos : PointPos .
eq next(AA,PPos) = AB                                   .
eq next(AB,normal) = AC . eq next(AB,reverse) = BC      .
\end{lstlisting}

The various tables (clear and release tables for the scheme plan, the
tables of the RBC) are encoded by defining a function for each
column. A typical example is the ``Clear Tracks''
column\footnote{Compared to the given control table, we add {\tt
RouteName4} to cover the exit track.} of the control table in
Figure \ref{fig:station}:
%
\begin{lstlisting}[columns=fixed]
op clearTracks : RouteName -> SetOfTracks  .
eq clearTracks(RouteName1A) = (AA, AB, AC) . 
...
eq clearTracks(RouteName4)  = empty        .
\end{lstlisting}
