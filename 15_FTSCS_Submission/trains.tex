The \verb|Train| class is the only time dependent entity in our
model. It is designed as an automaton with four
states \verb|stop|, \verb|acc| for accelerating, \verb|cons| for
constant speed, and \verb|brake|. There are transitions \verb|stop|
$\to $ \verb|acc| $\to $ \verb|cons| $\to $ \verb|brake|,
and \verb|acc| $\to$ \verb|brake| and vice versa. In addition, it has
fields representing the current distance (relative to a given
reference point 0), speed, acceleration, movement authority (relative
to 0), maximum speed, and the current track segment.
\begin{lstlisting}
class Train | state : TrainState, dist : NNegRat,
              speed : NNegRat, ac : NNegRat, ma : NNegRat,
              tseg : Track, maxspeed : NNegRat .
\end{lstlisting}
We assume that acceleration is linear, and -- apart from Scenario 3 in Section~\ref{sec:errorInjection} -- use a value of 1 for both acceleration and deceleration. Trains move according to Newton's laws, i.e., if at time $0$ a train is at \verb|DT| with speed \verb|S| and acceleration \verb|A|, then the speed at time \verb|R| is \verb|S + A*R| and the location is \verb|DT + S*R + A*R*R/2|. Its braking distance \verb|bd(S,A)| is \verb|S*S/2*A|. We show the rule for a train in the accelerating state.

\begin{lstlisting}
crl [acc] :
 < O1 : Inter | pointPositions : PointSettings >
 delta(< O : Train | state : acc, dist : DT, speed : S,
         ac : A, ma : MA, tseg : AN, maxspeed : MAX >, R)
      =>
      < O1 : Inter | pointPositions : PointSettings >
      trackseg(PointSettings, < O : Train |  
      state : if (S + A * R == MAX)     
              then cons
	      else (if R == mteMA(DT,S,A,MA)
	            then brake
	            else acc fi) fi, 
      dist : DT + S * R + R * R * A * (1/2),
      speed : S + A * R > ) if not AN == Exit .
\end{lstlisting}
The rule computes the new configuration of a train after
time \texttt{R} from its old configuration and the interlocking.
It is sufficient
to list those attributes that are updated, here speed,
location, and, possibly, the state.  The operator \verb|trackseg|
takes the new location of the train and the \verb|PointSettings| from
the interlocking and returns a new train object.  In the case that the
train has entered a new track it will update the train object
accordingly. Here, we combine the delta rule together with a state
transition, allowing us to exactly determine when a state transition
occurs. An alternative approach would be to decouple these 
orthogonal concepts by expressing the rule as equation + rules.
This, in turn, may lead to improvements when model checking.

The time \verb|R| is determined by the maximal time elapse which is,
in the acceleration state, the minimum of the following three
cases. 1) maximum speed is reached, 2) the end of a track segment is
reached, 3) the distance to the movement authority is not greater than
the required braking distance.
\begin{lstlisting}
ceq mte (< O : Train | state : acc,  dist : DT, speed : S, 
           ac : A, ma : MA , tseg : AN, maxspeed : MAX >)
           = min((MAX monus S) / A,
                ((endof(AN) + 1) monus DT) / S,
                mteMA(DT,S,A,MA))  if S > 0 .
eq mteMA(DT, S, A, MA) = (((MA monus 1) monus DT) monus 
                         (S * S / (2 * A))) / ( 2 * S) .
\end{lstlisting}
In case 1) we used \texttt{monos} for the maximum of the difference between two numbers and \texttt{0}. 
For cases 2) and 3) the calculation of \texttt{mte} involves quadratic
equations.  From \texttt{DT + S*R + A*R*R/2 < endof(AN)+1} we
could determine \texttt{R} using an approximation via Newton's
method. However, since, thanks to our default tick, we have \texttt{0 < R <= 1},
and therefore \texttt{0 < A*R*R/2 <= A*R/2},
we approximate the quadratic term either from below or from
above depending on the context: in the case of entering a new track we
ignore the quadratic term, and put the sampling point slightly late,
as we want to be on the new track already; in the case of calculating
where to start braking, we bring the event slightly forward,
i.e., we start braking slightly too early.
Both approximations are justified by the default tick.
%Finally, to keep the size of rational numbers under
%control, we determine the sampling time point with a precision of
%\texttt{1/1000}.


%Also note that each train system will build in error
%margins in case of braking, which are larger than the ones used here,
%and never brake exactly to the point (Cf also section on validation).
%Uli's formulation:
%An important question is whether our modelling is complete, that is, whether
%it is able to detect any possible error (???). The criteria for completeness
%given in [2] appear to be satisfied by our modelling except for the conditions
%... which are only approximately satisfied due to rounding in connection with
%the square root function. We conjecture that an appropriate weakening of
%the conditions ... still suffices to imply completeness, but leave this 
%to further work. The property of being non-zeno, required in [2], can be 
%seen for our system as follows. ...



